\documentclass[hyperref=unicode,presentation,10pt]{beamer}

\usepackage[absolute,overlay]{textpos}
\usepackage{array}
\usepackage{graphicx}
\usepackage{adjustbox}
\usepackage[version=4]{mhchem}
\usepackage{chemfig}
\usepackage{caption}

%dělení slov
\usepackage{ragged2e}
\let\raggedright=\RaggedRight
%konec dělení slov

\addtobeamertemplate{frametitle}{
	\let\insertframetitle\insertsectionhead}{}
\addtobeamertemplate{frametitle}{
	\let\insertframesubtitle\insertsubsectionhead}{}

\makeatletter
\CheckCommand*\beamer@checkframetitle{\@ifnextchar\bgroup\beamer@inlineframetitle{}}
\renewcommand*\beamer@checkframetitle{\global\let\beamer@frametitle\relax\@ifnextchar\bgroup\beamer@inlineframetitle{}}
\makeatother
\setbeamercolor{section in toc}{fg=red}
\setbeamertemplate{section in toc shaded}[default][100]

\usepackage{fontspec}
\usepackage{unicode-math}

\usepackage{polyglossia}
\setdefaultlanguage{czech}

\def\uv#1{„#1“}

\mode<presentation>{\usetheme{default}}
\usecolortheme{crane}

\setbeamertemplate{footline}[frame number]

\title[Crisis]
{C2062 -- Anorganická chemie II}

\subtitle{(Kyslík, síra,) selen, tellur, polonium a livermorium}
\author{Zdeněk Moravec, hugo@chemi.muni.cz \\ \adjincludegraphics[height=60mm]{img/IUPAC_PSP.jpg}}
\date{}

\begin{document}

\begin{frame}
	\titlepage
\end{frame}

\section{Kyslík}
\frame{
	\frametitle{}
	\begin{columns}
		\begin{column}{0.5\textwidth}
			\vfill
			\begin{itemize}
				\item Značka O, protonové číslo 8.
				\item Plyn, nezbytný pro život na Zemi.
				\item Vytváří dvouatomové, paramagnetické molekuly \ce{O2}.\footnote[frame]{\href{https://www.youtube.com/watch?v=2Q7Co5EvYBU}{Magnetism of liquid nitrogen vs. liquid oxygen}}
				\item Paramagnetismus je u~kyslíku způsoben přítomností nepárových elektronů v MO.
				\item Molekulární kyslík je poměrně reaktivní, silné oxidační činidlo.
				\item Tvoří 21~\% objemových procent zemské atmosféry a 90~\% hmotnosti oceánů.
			\end{itemize}
			\vfill
		\end{column}
		\begin{column}{0.5\textwidth}
			\begin{figure}
				\adjincludegraphics[width=.83\textwidth]{img/Liquid_oxygen.jpg}
				\caption*{Kapalný kyslík.\footnote[frame]{Zdroj: \href{https://commons.wikimedia.org/wiki/File:Liquid_oxygen_in_a_beaker_(cropped_and_retouched).jpg}{Staff Sgt. Nika Glover, U.S. Air Force/Commons}}}
			\end{figure}
		\end{column}
	\end{columns}
}

\frame{
	\frametitle{}
	\begin{figure}
		\adjincludegraphics[width=.68\textwidth]{img/Diagramme_OM_O2.jpg}
		\caption*{Diagram molekulových orbitalů v molekule \ce{O2}.\footnote[frame]{Zdroj: \href{https://commons.wikimedia.org/wiki/File:Diagramme_OM_O2.jpg}{jflm/Commons}}}
	\end{figure}
}

\frame{
	\frametitle{}
	\begin{columns}
		\begin{column}{0.6\textwidth}
			\vfill
			\begin{itemize}
				\item Kapalný kyslík (Liquid OXygen -- LOX; $-$183~$^\circ$C) se společně s dusíkem vyrábí destilací zkapalněného vzduchu.
				\item Vzduch se zkapalňuje stlačením a~následnou isentropickou expanzí.
				\item Dříve se využívalo Joule-Thompsonova efektu (Carl von Linde), ale tento postup je málo účinný.
				\item Kapalná fáze je bohatější na kyslík a~plynná fáze pak na dusík. Frakční destilací pak můžeme separovat kromě dusíku i argon.
				\item Kapalný kyslík se využívá jako zdroj kyslíku pro nemocnice a další aplikace.
				\item Je také součástí raketových paliv.
			\end{itemize}
			\vfill
		\end{column}
		\begin{column}{0.4\textwidth}
			\begin{figure}
				\adjincludegraphics[width=\textwidth]{img/Delta_IV_launch.jpg}
				\caption*{Start rakety Delta IV.\footnote[frame]{Zdroj: \href{https://www.af.mil/News/Photos.aspx?igphoto=2000709622}{U.S. Air Force/Joe Davila}}}
			\end{figure}
		\end{column}
	\end{columns}
}

\frame{
	\frametitle{}
	\begin{itemize}
		\item \textit{Ozon} -- allotropní modifikace kyslíku, \ce{O3}.
		\item Vysoce reaktivní, modrý plyn. Je tvořen lomenými molekulami (116,8$^\circ$).
		\item Vyrábí se působením elektrického výboje nebo UV záření na kyslík.
		\item V chemii se využívá k oxidaci, zejména k přípravě organických peroxidů.
		\item Je to také velmi dobré desinfekční činidlo, má velmi silné baktericidní účinky.
		\item Přízemní ozón je pro člověka nebezpečný.
		\item Naopak ve stratosféře je ozón nezbytný pro existenci života na Zemi.
	\end{itemize}
	\begin{figure}
		\adjincludegraphics[width=\textwidth]{img/Ozon_Grenzstruktur.png}
	\end{figure}
}

\frame{
	\frametitle{}
	\begin{figure}
		\adjincludegraphics[width=.8\textwidth]{img/Uars_ozone_waves.jpg}
		\caption*{Ozonová vrstva v roce 1984 a 1997.\footnote[frame]{Zdroj: \href{https://commons.wikimedia.org/wiki/File:Uars_ozone_waves.jpg}{NASA/Commons}}}
	\end{figure}
}

\frame{
	\frametitle{}
	\begin{itemize}
		\item V chemických sloučeninách má kyslík zpravidla oxidační číslo $-$II.
		\item Výjimkou jsou peroxidy, hyperoxidy (superoxidy) a ozonidy.
	\end{itemize}
	\begin{center}
		\begin{tabular}{|l|l|l|}
			\hline
			Oxid & \ce{O^{2-}} & \ce{CO2}, \ce{K2O} \\\hline
			Peroxid & \ce{O$_2^{2-}$} & \ce{H2O2}, \ce{BaO2} \\\hline
			Hyperoxid & \ce{O$_2^{-}$} & \ce{NaO2}, \ce{KO2} \\\hline
			Ozonid & \ce{O$_3^{-}$} & \ce{NaO3}, \ce{KO3} \\\hline
		\end{tabular}
		\begin{itemize}
			\item Známe také sloučeniny, obsahující kyslík v oxidačním stavu +2:
			\begin{itemize}
				\item Difluorid kyslíku, \ce{OF2} -- připravuje se reakcí fluoru se zředěným hydroxidem sodným:
				\item \ce{2 F2 + 2 NaOH -> OF2 + 2 NaF + H2O}
				\item Difluorid dikyslíku, \ce{O2F2} -- vzniká přímou reakcí kyslíku s fluorem v přítomnosti elektrického výboje.
			\end{itemize}
			\item Koordinační číslo kyslíku je zpravidla 2, ale může dosáhnout až hodnoty 8 u oxidů alkalických kovů, např. \ce{Li2O} nebo \ce{Na2O}.
		\end{itemize}
	\end{center}
}

\frame{
	\frametitle{}
	\begin{figure}
		\adjincludegraphics[width=.85\textwidth]{img/CaF2_polyhedra.png}
		\caption*{Krystalová struktura \ce{Li2O}, šedé atomy jsou kyslíky.\footnote[frame]{Zdroj: \href{https://commons.wikimedia.org/wiki/File:CaF2_polyhedra.png}{Solid State/Commons}}}
	\end{figure}
}

%Koordinační číslo 8; Li2O, Na2O

\section{Síra}
\frame{
	\frametitle{}
	\begin{itemize}
		\item Značka S, protonové číslo 16. Pevná, žlutá látka.
		\item Vytváří cyklické osmiatomové molekuly, \ce{S8}. Známe několik dalších allotropických modifikací.\footnote[frame]{\href{https://chem.libretexts.org/Bookshelves/Inorganic_Chemistry/Map\%3A_Inorganic_Chemistry_(Housecroft)/16\%3A_The_Group_16_Elements/16.04\%3A_The_Elements/16.4C\%3A_Sulfur_-_Allotropes}{Sulfur - Allotropes}}
		\begin{itemize}
			\item Nejstabilnější je kosočtverečná modifikace $\alpha$.
			\item Při teplotě 95,3~$^\circ$C přechází na jednoklonnou modifikaci $\beta$.
			\item Jednoklonná modifikace $\gamma$ vzniká pomalým ochlazováním taveniny síry.
		\end{itemize}
		\item Síra se těží Fraschovým způsobem (dnes už jen minoritně) nebo povrchově.
		\item Hlavními zdroji jsou ropa (Clausův proces) a zemní plyn, kde je síra ve formě sulfanu.\footnote[frame]{\href{https://www.usgs.gov/centers/nmic/sulfur-statistics-and-information}{Sulfur Statistics and Information}}
		\item Jednou z průmyslově nejvýznamnějších sloučenin síry je kyselina sírová, \ce{H2SO4}. Mimo ní známe velké množství dalších kyselin obsahujících síru.
		\item \ce{SO2} pro výrobu kyseliny sírové se získává při zpracování sulfidových rud.
	\end{itemize}
}

\frame{
	\frametitle{}
	\vfill
	\begin{columns}
		\begin{column}{0.5\textwidth}
			\begin{figure}
				\adjincludegraphics[width=\textwidth]{img/Sulfur-sample.jpg}
				\caption*{Síra.\footnote[frame]{Zdroj: \href{https://commons.wikimedia.org/wiki/File:Sulfur-sample.jpg}{Ben Mills/Commons}}}
			\end{figure}
		\end{column}
		\begin{column}{0.5\textwidth}
			\begin{figure}
				\adjincludegraphics[width=.8\textwidth]{img/Cyclooctasulfur.png}
				\caption*{Struktura cyklooktasíry, \ce{S8}.\footnote[frame]{Zdroj: \href{https://commons.wikimedia.org/wiki/File:Cyclooctasulfur-above-3D-balls.png}{Benjah-bmm27/Commons}}}
			\end{figure}
		\end{column}
	\end{columns}
	\vfill
}

\frame{
	\frametitle{}
	\begin{figure}
		\adjincludegraphics[width=\textwidth]{img/Frasch_Process.png}
		\caption*{Těžba síry Fraschovým procesem\footnote[frame]{\href{https://commons.wikimedia.org/wiki/File:Frasch_Process_used_on_a_salt_dome.png}{Zdroj: Joyce A. Ober/Commons}}}
	\end{figure}
}

\frame{
	\frametitle{}
	\begin{figure}
		\adjincludegraphics[height=.7\textheight]{img/Sulfur-Miner.jpg}
		\caption*{Síra se těží i povrchově, těžař s 90 kg nákladem síry z vulkánu \href{https://commons.wikimedia.org/wiki/File:Kawah-Ijen_Indonesia_Ijen-Sulfur-Miner-04.jpg}{\textit{Ijen v Indonésii}}.\footnote[frame]{Zdroj: \href{https://commons.wikimedia.org/wiki/File:Kawah-Ijen_Indonesia_Ijen-Sulfur-Miner-04.jpg}{CEphoto, Uwe Aranas}}}
	\end{figure}
}

\frame{
	\frametitle{}
	\begin{itemize}
		\item Síra se využívá k \textit{vulkanizaci kaučuku}, tzn. k zesíťování řetězců přírodního kaučuku za vzniku \textit{pryže}.\footnote[frame]{\href{https://www.gumex.cz/blog/vznik-a-vyvoj-gumy-118}{Vznik a vývoj gumy}}
		\item Mezi řetězci vznikají polysulfidové můstky, které se váží na nenasycené uhlíky řetězců.
	\end{itemize}
	\begin{figure}
		\adjincludegraphics[width=.8\textwidth]{img/vulcanization.png}
	\end{figure}
}

\subsection{Kyseliny síry}
\frame{
	\frametitle{}
	\begin{figure}
		\adjincludegraphics[height=.8\textheight]{img/S-acids.png}
	\end{figure}
}

\frame{
	\frametitle{}
	\begin{itemize}
		\item Kyselina sírová se vyrábí oxidací \ce{SO2} na \ce{SO3}.
		\item \ce{SO2} lze získat spalováním elementární síry, dnes se ale často využívá oxid siřičitý z redukce sulfidických rud a jiných chemických procesů:
		\item \ce{2 Cu2O + Cu2S -> 6 Cu + SO2}
		\item Oxid siřičitý lze pak oxidovat buď katalyticky (kontaktní způsob):
		\item \ce{2 SO2 + O2 ->[V2O5 + K2SO4] 2 SO3}
		\item nebo komorovým (dnes už nepoužívaným) způsobem:
		\item \ce{SO2 + NO2 -> SO3 + NO}
		\item Získaný oxid sírový je jímán do kyseliny sírové, čímž vzniká kyselina disírová a posléze \textit{oleum}.
		\item Oleum je následně ředěno vodou zpět na kyselinu sírovou.
		\item Přímé pohlcování oxidu sírového do vody je nevýhodné, jelikož vzniká obtížně kondenzovatelný aerosol.
	\end{itemize}
}

\subsection{Oxid sírový}
\frame{
	\frametitle{}
	\begin{itemize}
		\item Oxid sírový, \ce{SO3}, je bezbarvá olejovitá kapalina nebo bílá krystalická látka.
		\item Teplota tání 17~$^\circ$C a varu 45~$^\circ$C.
		\item V plynném stavu existuje jako monomer, v souladu s teorií VSEPR má trojúhelníkovou geometrii (D$_{3h}$).
		\item V kapalném a pevném stavu existuje rovnováha mezi monomerní a trimerní formou.
		\item Trimer, \ce{S3O9} má vaničkovou konformaci.
		\begin{figure}
			\adjincludegraphics[height=.3\textheight]{img/Sulfur-trioxide-trimer.png}
			\caption*{Krystalová struktura trimeru oxidu sírového.\footnote[frame]{Zdroj: \href{https://commons.wikimedia.org/wiki/File:Sulfur-trioxide-trimer-from-xtal-1967-3D-balls-B.png}{Ben Mills/Commons}}}
		\end{figure}
	\end{itemize}
}

\frame{
	\frametitle{}
	\begin{figure}
		\adjincludegraphics[height=.65\textheight]{img/Schefeltrioxid.jpg}
		\caption*{Vzorek trimeru oxidu sírového.\footnote[frame]{Zdroj: \href{https://commons.wikimedia.org/wiki/File:Schefeltrioxid.jpg}{H. Hiller/Commons}}}
	\end{figure}
}

\section{Úvod -- selen, tellur, polonium a livermorium}
\frame{
	\frametitle{}
	\vfill
	\begin{tabular}{|c|l|l|l|}
	\hline
	 & \textit{Selen} & \textit{Tellur} & \textit{Polonium} \\\hline
	 El. konfigurace & 3d$^{10}$ 4s$^{2}$ 4p$^{4}$ & 4d$^{10}$ 5s$^{2}$ 5p$^{4}$ & 4f$^{14}$ 5d$^{10}$ 6s$^{2}$ 6p$^{4}$ \\\hline
	 Teplota tání [$^\circ$C] & 221 & 449,51 & 254 \\\hline
	 Teplota varu [$^\circ$C]  & 685 & 987,85 & 962 \\\hline
	 Objeven & 1817 & 1782 & 1898 \\\hline
	 Vzhled & šedý nebo červený\footnote[frame]{Zdroj: \href{https://commons.wikimedia.org/wiki/File:SeBlackRed.jpg}{W. Oelen/Commons}} & stříbrnošedý\footnote[frame]{Zdroj: \href{https://commons.wikimedia.org/wiki/File:Tellurium2.jpg}{Materialscientist/Commons}} & stříbrný\footnote[frame]{Zdroj: \href{https://en.wikipedia.org/wiki/File:Polonium.jpg}{Ralph E. Lapp/Wikipedia}} \\
	 &  \begin{minipage}{.2\textwidth}
	 	\adjincludegraphics[width=\linewidth]{img/SeBlackRed.jpg}
	 \end{minipage}
	 	& \begin{minipage}{.2\textwidth}
	 		\adjincludegraphics[width=\linewidth]{img/Tellurium2.jpg}
	 	\end{minipage} & \begin{minipage}{.2\textwidth}
	 	\adjincludegraphics[width=\linewidth]{img/Polonium.jpg}
 	\end{minipage} \\\hline
	\end{tabular}
	\vfill
}

\frame{
	\frametitle{}
	\vfill
		\textbf{Livermorium}
		\begin{columns}
			\begin{column}{.7\textwidth}
				\begin{itemize}
					\item Umělý prvek, protonové číslo 116, Lv.\footnote[frame]{\href{https://www.rsc.org/periodic-table/element/116/livermorium}{Livermorium -- rsc.org}}
					\item Poprvé byl připraven v roce 2000 v Dubně.\footnote[frame]{\href{https://journals.aps.org/prc/abstract/10.1103/PhysRevC.69.021601}{Observation of the decay of \ce{^{292}116}}}
					\item \ce{^{248}_{96}Cm + ^{48}_{20}Ca -> ^{296}_{116}Lv^* -> ^{293}_{116}Lv + 3 ^1_0n}
					\item Nejstabilnějším izotopem je \ce{^{293}Lv} s poločasem rozpadu 70~ms.
					\item \ce{^{293}Lv ->[$T_{\frac{1}{2}}$ = 70 ms] ^{289}Fl + $\alpha$}
					\item Prvek byl v roce 2011 pojmenován \emph{Livermorium} na počest sídla Lawrence Livermore National Laboratory.\footnote[frame]{\href{https://old.iupac.org/publications/ci/2012/3404/iw1_periodic_table.html}{Flerovium and Livermorium Join the Periodic Table}}
				\end{itemize}
			\end{column}

			\begin{column}{.3\textwidth}
				\begin{tabular}{|l|l|}
					\hline
					Izotop & T$_{1/2}$ [ms] \\\hline
					$^{290}$Lv & 9 \\\hline
					$^{291}$Lv & 26 \\\hline
					$^{292}$Lv & 16 \\\hline
					$^{293}$Lv & 70 \\\hline
					$^{293m}$Lv & 80 \\\hline
				\end{tabular}
			\end{column}
		\end{columns}
	\vfill
}

\frame{
	\frametitle{}
	\vfill
	\begin{figure}
		\adjincludegraphics[height=.75\textheight]{img/LLNL_Aerial_View.jpg}
		\caption*{Areál Lawrence Livermore National Laboratory\footnote[frame]{\href{https://commons.wikimedia.org/wiki/File:LLNL_Aerial_View.jpg}{Zdroj: llnl.gov/Commons}}}
	\end{figure}
	\vfill
}

\section{Historie}
\subsection{Selen, tellur}
\frame{
	\frametitle{}
	\vfill
	\begin{itemize}
		\item Selen byl objeven roku 1817 chemiky Jönsem J. Berzeliem a Johanem Gottliebem Gahnem ve zbytcích po spalování síry.
		\item Tellur byl objeven už roku 1782 v rudách z Transylvánie (Rumunska). Izoloval jej rakouský chemik J. F. Müller von Reichenstein.
		\item Název tellur (z latinského \textit{tellus} -- země) mu dal rakouský chemik M. H. Klaproth.
	\end{itemize}
	\begin{columns}
		\begin{column}{.3\textwidth}
			\begin{figure}
				\adjincludegraphics[height=.3\textheight]{img/Franz-Joseph_Müller_von_Reichenstein.jpg}
				\caption*{J. F. Müller}
			\end{figure}
		\end{column}

	\begin{column}{.3\textwidth}
		\begin{figure}
			\adjincludegraphics[height=.3\textheight]{img/Jöns_Jacob_Berzelius.jpg}
			\caption*{Jöns Jacob Berzelius}
		\end{figure}
	\end{column}

	\begin{column}{.3\textwidth}
		\begin{figure}
			\adjincludegraphics[height=.3\textheight]{img/Gahn_Johan_Gottlieb.jpg}
			\caption*{Johan Gottlieb Gahn}
		\end{figure}
	\end{column}
	\end{columns}
	\vfill
}

\subsection{Objev polonia}
\frame{
	\frametitle{}
	\vfill
	\begin{itemize}
		\item Polonium bylo objeveno v roce 1898 Marií a Pierem Curie izolací z \textit{uraninitu}.\footnote[frame]{\href{http://web.lemoyne.edu/~giunta/curiespo.html}{On a New Radioactive Substance Contained in Pitchblende}}
		\item Zjistili, že po extrakci uranu a thoria byla radioaktivita zbylého materiálu vyšší než úhrnná radioaktivita těchto dvou prvků.
		\item Polonium se rozpadá mechanismem $\alpha$ s poločasem rozpadu 138,4~dne.
		\item \ce{^{210}_{ 84}Po -> ^{206}_{ 82}Pb + ^4_2He}
	\end{itemize}
	\begin{columns}
		\begin{column}{.2\textwidth}

		\end{column}
		\begin{column}{.3\textwidth}
			\begin{figure}
				\adjincludegraphics[height=0.3\textheight]{img/Pierre_Curie_et_Marie_Sklodowska_Curie_1895.jpg}
				\caption*{Marie a Piere Curie.}
			\end{figure}
		\end{column}
		\begin{column}{.3\textwidth}
			\begin{figure}
				\adjincludegraphics[height=0.3\textheight]{img/Uraninite-usa32abg.jpg}
				\caption*{Uraninit.\footnote[frame]{Zdroj: \href{https://commons.wikimedia.org/wiki/File:Uraninite-usa32abg.jpg}{Robert M. Lavinsky/Commons}}}
			\end{figure}
		\end{column}
		\begin{column}{.2\textwidth}

		\end{column}
	\end{columns}
	\vfill
}

\frame{
	\frametitle{}
	\vfill
	\begin{columns}
		\begin{column}{.75\textwidth}
			\begin{itemize}
				\item Marie Curie Sklodowská se narodila 7. listopadu 1867 ve Varšavě.\footnote[frame]{\href{http://www.converter.cz/fyzici/curie-sklodowska.htm}{Marie Curie-Sklodowská}}
				\item Roku 1891 odešla do Paříže, studovat na Sorbonu.
				\item 1895 se vdala za profesora fyziky Pierra Curie.
				\item 1898 Objevila nové prvky polonium a radium.
				\item 1900 Stala se první ženou na Ecole Normale Superieure.
				\item 1903 získala Nobelovu cenu za fyziku za zkoumání radiačních dějů.
				\item 1911 získala Nobelovu cenu za chemii za objev radia a polonia.
				\item Zemřela 4. července 1934 ve Francii na chorobu vyvolanou účinky ionizujícího záření.
			\end{itemize}
		\end{column}

		\begin{column}{.3\textwidth}
			\begin{figure}
				\adjincludegraphics[height=\textwidth]{img/Marie_Curie_c._1920s.jpg}
				\caption*{Marie Curie-Skłodowská.\footnote[frame]{Zdroj: \href{https://commons.wikimedia.org/wiki/File:Marie_Curie_c._1920s.jpg}{Henri Manuel/Commons}}}
			\end{figure}
		\end{column}
	\end{columns}
	\vfill
}

\section{Chemické a fyzikální vlastnosti}
\frame{
	\frametitle{}
	\vfill
	\begin{itemize}
		\item Selen má šest stabilních izotopů, tellur osm. Kvůli tomu není možné určit atomovou hmotnost prvků s vyšší přesností.
		\item Kyslík a síra jsou izolanty, selen a tellur jsou polovodiče, polonium je vodič (kov).
		\item S rostoucím kovovým charakterem prvku roste i bazicita. Selen s~kyselinou chlorovodíkovou reaguje velmi pomalu, naproti tomu tellur se v ní rozpouští a polonium vytváří růžové roztoky \ce{Po^{II}} solí.
		\item S elektropozitivními prvky vytvářejí chalkogenidy -- selenidy, telluridy a polonidy.
		\item Ve sloučeninách s elektronegativními prvky mohou dosahovat oxidačních čísel II, IV a VI.
		\item Teplotní stabilita hydridů klesá v řadě: \item \ce{H2O} $>$ \ce{H2S} $>$ \ce{H2Se} $>$ \ce{H2Te} $>$ \ce{H2Po}
		\item Sloučeniny selenu, telluru i polonia jsou toxické, těkavé sloučeniny, řadí se mezi velmi nebezpečné.
	\end{itemize}
	\vfill
}

\frame{
	\frametitle{}
	\vfill
	\begin{itemize}
		\item Selen i tellur tvoří několik allotropních modifikacích.
		\item \textit{Červený selen} krystaluje ve třech rozdílných polymorfních modifikacích, tvořených cyklickými molekulami \ce{Se8}, které jsou analogické cyklooktasíře.
		\begin{itemize}
			\item $\alpha$-\ce{Se8} vzniká pomalým odpařením roztoku černého selenu v \ce{CS2}.
			\item $\beta$-\ce{Se8} vzniká rychlým odpařením roztoku černého selenu v \ce{CS2}.
			\item $\gamma$-\ce{Se8} vzniká solovolýzou roztoku dipiperidinotetraselanu v sirouhlíku:\footnote[frame]{\href{https://doi.org/10.1039/C39770000834}{X-Ray crystal structure of a new red, monoclinic form of cyclo-octaselenium, \ce{Se8}}}
			\item \ce{8 Se4(NC5H10)2 + 16 CS2 -> 3 Se8 + 8 Se(S2CNC5H10)2}
		\end{itemize}
		\item \textit{Červený amorfní selen} vzniká kondenzací par nebo redukcí kyseliny selenové.
		\item \textit{Šedý selen} je termodynamicky nejstabilnější, má kovové vlastnosti a je fotovodivý. Vzniká zahřátím jiné modifikace selenu nebo pomalou krystalizací taveniny. Krystaluje v hexagonální soustavě, tvoří jej šroubovicovité řetězce.
		\item \textit{Černý selen} se připravuje prudkým ochlazením taveniny, má sklovitý charakter. Nad teplotou 50~$^\circ$C měkne a při teplotě 180~$^\circ$C přechází na šedou, hexagonální formu.
	\end{itemize}
	\vfill
}

\frame{
	\frametitle{}
	\vfill
	\begin{figure}
		\adjincludegraphics[height=0.5\textheight]{img/Selen_1.jpg}
		\caption*{Černý, šedý a červený selen.\footnote[frame]{Zdroj: \href{https://commons.wikimedia.org/wiki/File:Selen_1.jpg}{Tomihahndorf/Commons}}}
	\end{figure}
	\vfill
}

\frame{
	\frametitle{}
	\vfill
	\begin{figure}
		\adjincludegraphics[height=0.7\textheight]{img/Black_Selenium.jpg}
		\caption*{Černý selen.\footnote[frame]{Zdroj: \href{https://commons.wikimedia.org/wiki/File:Black_Selenium.jpg}{Best Sci-Fatcs/Commons}}}
	\end{figure}
	\vfill
}

\frame{
	\frametitle{}
	\vfill
	\begin{itemize}
		\item Tellur vytváří krystalickou a amorfní modifikaci.
		\item Krystalický tellur je tvořen sítí spirálovitých řetězců, podobně jako šedý selen. Má stříbro-bílou barvu a je kovové lesklý.
		\item Je polovodivý a vykazuje i fotovodivé vlastnosti.
		\item Amorfní modifikace je hnědočerný prášek, připravuje se srážením roztoků kyseliny telluričité nebo hexahydrogentellurové.
	\end{itemize}
	\begin{figure}
		\adjincludegraphics[height=0.4\textheight]{img/Tellurium-acrylic.jpg}
		\caption*{Kovový tellur zalitý v akrylu.\footnote[frame]{Zdroj: \href{https://commons.wikimedia.org/wiki/File:An_acrylic_cube_specially_prepared_for_element_collectors_containing_a_sample_of_pure_silicon_tellurium.JPG}{Rasiel Suarez on behalf of Luciteria LLC/Commons}}}
	\end{figure}
	\vfill
}

\frame{
	\frametitle{}
	\vfill
	\begin{itemize}
		\item Polonium existuje také ve dvou modifikacích.\footnote[frame]{\href{https://doi.org/10.1016/0022-1902(66)80270-1}{The structures of polonium and its compounds—I $\alpha$ and $\beta$ polonium metal}}
		\item $\alpha$-Po krystaluje v primitivní kubické mřížce. Polonium je jediný prvek, který tvoří tento typ mřížky.
		\item $\beta$-Po krystaluje v kosočtverečné soustavě.
		\item $\alpha$-Po přechází na $\beta$-Po při teplotě 36~$^\circ$C.
		\item Přesnou teplotu přechodu je obtížné určit, polonium se vlivem\\ radioaktivního rozpadu samovolně zahřívá.
	\end{itemize}
	\begin{columns}
		\begin{column}{.6\textwidth}
			\begin{figure}
				\adjincludegraphics[height=0.35\textheight]{img/Alpha_po_lattice.jpg}
				\caption*{Krystalová mřížka polonia.\footnote[frame]{Zdroj: \href{https://commons.wikimedia.org/wiki/File:Alpha_po_lattice.jpg}{Cadmium/Commons}}}
			\end{figure}
		\end{column}
		\begin{column}{.4\textwidth}
			\textbf{Hlavní izotopy polonia}
			\begin{tabular}{|l|r@{,}ll|}
				\hline
				\textbf{N} & \multicolumn{3}{c|}{\textbf{Poločas rozpadu}}
				\\\hline
				208 & 2 & 898 & let \\\hline
				209 & 125 & 2 & let \\\hline
				210 & 138 & 376 & dne \\\hline
			\end{tabular}
		\end{column}
	\end{columns}
	\vfill
}

\section{Výskyt a získávání prvků}
\subsection{Selen}
\frame{
	\frametitle{}
	\begin{columns}
		\begin{column}{.65\textwidth}
			\vfill
			\begin{itemize}
				\item Ryzí selen se vyskytuje jen velmi zřídka a často ve směsi se sírou nebo jinými prvky.\footnote[frame]{\href{https://www.mindat.org/element/Selenium}{The mineralogy of Selenium}}
				\item V minerálech se vyskytuje častěji ve formě sloučenin -- selenidů, seleničitanů a selenanů.
				\item Pozor na minerál \textit{selenit}, i přes matoucí název, jde o variantu sádrovce.\footnote[frame]{\href{https://www.mindat.org/min-5527.html}{Selenite}}
				\item Známe více než 100 minerálů obsahujících selen, z velké části obsahují tyto minerály i síru (sulfidy a sírany).
			\end{itemize}
			\vfill
		\end{column}

		\begin{column}{.4\textwidth}
			\begin{figure}
				\adjincludegraphics[width=\textwidth]{img/Selenium_in_sandstone.jpg}
				\caption*{Selen na pískovci.\footnote[frame]{Zdroj: \href{https://commons.wikimedia.org/wiki/File:Selenium_in_sandstone_Westwater_Canyon_Section_23_Mine_Grants,_New_Mexico.jpg}{James St. John/Commons}}}
			\end{figure}
		\end{column}
	\end{columns}
}

\subsection{Selen}
\frame{
	\frametitle{}
	\vfill
	\begin{itemize}
		\item Roční světová produkce selenu je odhadována na 3000 tun.\footnote[frame]{\href{https://doi.org/10.1007/s11015-010-9280-7}{Selenium and tellurium: state of the markets, the crisis, and its consequences}}
		\item Celosvětové zásoby jsou 80--90 000 tun.
		\item Mezi největší producenty patří Japonsko, USA a Čína.
		\item Hlavní surovinou pro výrobu selenu jsou anodové kaly po rafinaci mědi a odpadní kaly z výroby kyseliny sírové.
		\item Ty se praží na vzduchu s uhličitanem sodným a poté se louží vodou:
		\item \ce{M2Se + Na2CO3 + O2 ->[650 $^\circ$C] 2 M + Na2SeO3 + CO2}
		\item Výluh se neutralizuje kyselinou sírovou, vysráží se oxid telluričitý a ten se dále zpracovává na tellur.
		\item V roztoku zůstává seleničitan, který je následně redukován oxidem siřičitým:
		\item \ce{H2SeO3 + 2 SO2 + H2O -> Se + 2 H2SO4}
	\end{itemize}
	\vfill
}

\frame{
	\frametitle{}
	\vfill
	\begin{itemize}
		\item Čištění selenu můžeme provést odpařením selenu v oxidační atmosféře a následným pohlcení oxidu seleničitého do vody:\footnote[frame]{\href{https://doi.org/10.1002/14356007.a23_525}{Selenium and Selenium Compounds}}
		\item \ce{SeO2 + H2O -> H2SeO3}
		\item Čistý selen je pak redukován vysoce čistým oxidem siřičitým:
		\item \ce{H2SeO3 + 2 SO2 + H2O -> Se + 2 H2SO4}
		\item Vysoce čistý selen se připravuje zahříváním surového materiálu v atmosféře vodíku na teplotu 650~$^\circ$C za vzniku selanu.
		\item \ce{Se + H2 ->[650 $^\circ$C] H2Se}
		\item Ten se následně rozkládá v křemenné trubici při teplotě 1000~$^\circ$C.
		\item \ce{H2Se ->[1000 $^\circ$C] Se + H2}
		\item Výhodou tohoto postupu je, že jiné prvky, které se mohou v surovinách nacházet, netvoří hydridy stabilní za tak vysokých teplot.
	\end{itemize}
	\vfill
}

\subsection{Tellur}
\frame{
	\frametitle{}
	\vfill
	\begin{itemize}
		\item Tellur je poměrně vzácný, jeho koncentrace v zemské kůře je srovnatelná s platinou.
		\item Podobně jako selen se vzácně vyskytuje v ryzí formě, ale častěji je ve vázané formě jako tellurid a doprovází sulfidické rudy.
	\end{itemize}
	\begin{figure}
		\adjincludegraphics[height=0.5\textheight]{img/Tellurium-89043.jpg}
		\caption*{Tellur na sylvanitu (\ce{(Au,Ag)2Te2}).\footnote[frame]{Zdroj: \href{https://commons.wikimedia.org/wiki/File:Tellurium-89043.jpg}{Christian Rewitzer/Commons}}}
	\end{figure}
	\vfill
}

\frame{
	\frametitle{}
	\vfill
	\begin{itemize}
		\item \textbf{Sylvanit}
		\item \ce{(Au,Ag)2Te4}, monoklinický minerál. Patří mezi nejběžnější telluridy.\footnote[frame]{\href{http://geologie.vsb.cz/loziska/loziska/rudy/sylvanit.html}{Sylvanit}}
		\item Poměr Au:Ag je blízký 1:1.\footnote[frame]{\href{https://www.mindat.org/min-3849.html}{Sylvanite}}
	\end{itemize}
	\begin{columns}
		\begin{column}{.5\textwidth}
			\begin{figure}
				\adjincludegraphics[height=0.35\textheight]{img/Sylvanite.jpg}
				\caption*{Sylvanit, Rumunsko.\footnote[frame]{Zdroj: \href{https://commons.wikimedia.org/wiki/File:Sylvanite.jpg}{Didier Descouens/Commons}}}
			\end{figure}
		\end{column}

		\begin{column}{.5\textwidth}
			\begin{figure}
				\adjincludegraphics[height=0.35\textheight]{img/Gold-Sylvanite-282521.jpg}
				\caption*{Sylvanit se zlatem.\footnote[frame]{Zdroj: \href{https://commons.wikimedia.org/wiki/File:Gold-Sylvanite-282521.jpg}{Robert M. Lavinsky/Commons}}}
			\end{figure}
		\end{column}
	\end{columns}
	\vfill
}

\frame{
	\frametitle{}
	\vfill
	\begin{columns}
		\begin{column}{.65\textwidth}
			\begin{itemize}
				\item Výroba telluru je z velké části podobná výrobě selenu.
				\item Vysrážený oxid se převede na telluričitan:
				\item \ce{TeO2 + 2 NaOH -> Na2TeO3 + H2O}
				\item Ten je následně elektrolyticky redukován na tellur:\footnote[frame]{\href{https://doi.org/10.3389/fchem.2020.00084}{Electrochemical Mechanism of Tellurium Reduction in Alkaline Medium}}
				\item \ce{Na2TeO3 + H2O -> Te + 2 NaOH + O2}
				\item Nebo se postupuje podobně jako u selenu:
				\item \ce{TeO2 + 2 SO2 + 2 H2O -> Te + 2 H2SO4}
				\item Roční produkce telluru se pohybuje v rozmezí 400--700 tun.\footnote[frame]{\href{https://doi.org/10.1007/s11015-010-9280-7}{Selenium and tellurium: state of the markets, the crisis, and its consequences}} Hlavními producenty jsou Kanada, USA a Čína.
			\end{itemize}
		\end{column}

		\begin{column}{.4\textwidth}
			\begin{figure}
				\adjincludegraphics[height=0.33\textheight]{img/Tellurium_crystal.jpg}
				\caption*{Ultračistý tellur.\footnote[frame]{Zdroj: \href{https://commons.wikimedia.org/wiki/File:Tellurium_crystal.jpg}{Dschwen/Commons}}}
			\end{figure}
		\end{column}
	\end{columns}
	\vfill
}

\subsection{Polonium}
\frame{
	\frametitle{}
	\vfill
	\begin{columns}
		\begin{column}{.65\textwidth}
			\begin{itemize}
				\item Polonium je možné izolovat ze smolince (uraninitu), ale postup je zdlouhavý a obtížný. Jeho koncentrace je přibližně 0,1 mg/tunu rudy.
				\item \ce{^{210}Po} se připravuje v reaktoru bombardováním jádra \ce{^{209}Bi} neutrony.
				\item \ce{^{209}Bi + n -> ^{210}Bi -> ^{210}Po + e^-}
				\item Poločas rozpadu \ce{^{210}Po} je 138,38 dne.
				\item Výroba \ce{^{210}Po} probíhá v Rusku. Měsíčně se vyrobí asi 8 g, ty jsou pak lodí transportovány do USA.\footnote[frame]{\href{https://hps.org/documents/polonium_210_story.pdf}{Why \ce{^{210}Po}?}}
			\end{itemize}
		\end{column}

		\begin{column}{.4\textwidth}
			\begin{figure}
				\adjincludegraphics[width=\textwidth]{img/Pechblende_Pribram.jpg}
				\caption*{Uraninit, Příbram.\footnote[frame]{Zdroj: \href{https://commons.wikimedia.org/wiki/File:Pechblende_Pribram.jpg}{Weirdmeister/Commons}}}
			\end{figure}
		\end{column}
	\end{columns}


	\vfill
}

\section{Využití prvků}
\subsection{Selen}
\frame{
	\frametitle{}
	\vfill
	\begin{columns}
		\begin{column}{.6\textwidth}
			\textbf{Barvení skla}
				\begin{itemize}
					\item Majoritní využití selenu, spotřebuje až 50~\% produkce selenu.\footnote[frame]{\href{https://ceramics.onlinelibrary.wiley.com/doi/pdf/10.1111/j.1151-2916.1919.tb18751.x}{Production of selenium red glass}}
					\item V malých koncentracích způsobuje odbarvení skla.
					\item Ve vyšších koncentracích (1-2 kg na tunu skla) způsobuje zabarvení skla do růžové až červené barvy.\footnote[frame]{\href{https://www.diva-portal.org/smash/get/diva2:11081/FULLTEXT01.pdf}{Red Glass Coloration – A Colorimetric and Structural Study}}
					\item Společně se sulfidem kademnatým vytváří velmi kvalitní rubínově červenou barvu.
				\end{itemize}
		\end{column}
		\begin{column}{.4\textwidth}
		\begin{figure}
			\adjincludegraphics[width=\textwidth]{img/Vintage_cranberry_glass.jpg}
			\caption*{Sklo barvené selenem.\footnote[frame]{Zdroj: \href{https://commons.wikimedia.org/wiki/File:Vintage_cranberry_glass.jpg}{PetitPoulailler/Commons}}}
		\end{figure}
	\end{column}
	\end{columns}
	\vfill
}

\frame{
	\frametitle{}
	\vfill
	\begin{columns}
		\begin{column}{.6\textwidth}
			\textbf{Xerox}
			\begin{itemize}
				\item Suchý kopírovací proces.
				\item Byl vynalezen v letech 1937--1942.
				\item Využívá se polovodivá vrstva, na které se vytvoří latentní obraz a na něj se přichytí toner.
				\item Využívá se v kopírkách a laserových tiskárnách.
				\item Proces se skládá z několika kroků:
			\end{itemize}
			\begin{enumerate}
				\item Fotocitlivá vrstva (tenká vrstva amorfního selenu) je nejprve nabita působením elektrického pole (15~kV).
				\item Obraz předlohy se promítne na tuto vrstvu. Místa, na která světlo nedopadne zůstanou nabitá.
			\end{enumerate}
		\end{column}
		\begin{column}{.4\textwidth}
			\begin{figure}
				\adjincludegraphics[height=.65\textheight]{img/Xerographic_photocopy_process_cs.png}
				\caption*{Princip xeroxu.\footnote[frame]{Zdroj: \href{https://commons.wikimedia.org/wiki/File:Xerographic_photocopy_process_cs.svg}{Yzmo/Commons}}}
			\end{figure}
		\end{column}
	\end{columns}
	\vfill
}

\frame{
	\frametitle{}
	\vfill
	\begin{columns}
		\begin{column}{.65\textwidth}
			\begin{enumerate}
				\setcounter{enumi}{2}
				\item Toner se elektrostaticky přichytí na nabitou vrstvu. Zrna toneru mají velikost zhruba 10~$\mu$m a jsou triboelektricky (třením) nabita.
				\item Toner se na papír přenese nabitím papíru.
				\item Obraz je zafixován působením vyšší teploty a tlaku.
				\item Válec je mechanicky očištěn.
				\item Latentní obraz se vymaže působením silného světelného zdroje. Napětí klesne na cca 100~V.
			\end{enumerate}

			\begin{itemize}
				\item V posledních letech se amorfní selen používá i pro konstrukci plošných detektorů RTG záření pro biomedicínské aplikace.\footnote[frame]{\href{https://doi.org/10.1016/j.jnoncrysol.2004.08.070}{Amorphous selenium as an X-ray photoconductor}}
			\end{itemize}
		\end{column}
		\begin{column}{.4\textwidth}
			\begin{figure}
				\adjincludegraphics[height=.65\textheight]{img/Xerographic_photocopy_process_cs.png}
				\caption*{Princip xeroxu.\footnote[frame]{Zdroj: \href{https://commons.wikimedia.org/wiki/File:Xerographic_photocopy_process_cs.svg}{Yzmo/Commons}}}
			\end{figure}
		\end{column}
	\end{columns}
	\vfill
}

\frame{
	\frametitle{}
	\vfill
	\textbf{Selenové usměrňovače}
	\begin{itemize}
		\item Polovodičové usměrňovače používané v~napájecích zdrojích s větším odběrem.
		\item Byly tvořeny soustavou hliníkových nebo ocelových destiček potažených tenkou vrstvou bismutu nebo niklu.
		\item Na povrchu destiček byla tenká vrstva (50-60~$\mu$m) selenu dopovaného halogeny.
		\item Každá destička zvládne asi 20 V závěrného napětí.
		\item Tyto usměrňovače byly postupně nahrazeny křemíkovými diodami, které jsou levnější, mají větší životnost a~menší úbytek napětí.
		\item Při přetížení se z nich uvolňuje silně zapáchající a toxický selan,\footnote[frame]{\href{https://pubchem.ncbi.nlm.nih.gov/compound/533}{Hydrogen selenide}} \ce{H2Se}.
	\end{itemize}
	\vfill
}

\frame{
	\frametitle{}
	\vfill
	\begin{columns}
		\begin{column}{.5\textwidth}
			\begin{figure}
				\adjincludegraphics[width=\textwidth]{img/Structure_selenium_rectifier.png}
				\caption*{Struktura selenového usměrňovače.\footnote[frame]{Zdroj: \href{https://commons.wikimedia.org/wiki/File:Structure_selenium_rectifier.svg}{Stündle/Commons}}}
			\end{figure}
		\end{column}
		\begin{column}{.5\textwidth}
			\begin{figure}
				\adjincludegraphics[width=.7\textwidth]{img/Selenium_Rectifier.jpg}
				\caption*{Osmideskový selenový usměrňovač, 160~V, 450~mA.\footnote[frame]{Zdroj: \href{https://commons.wikimedia.org/wiki/File:Selenium_Rectifier.jpg}{Binarysequence/Commons}}}
			\end{figure}
		\end{column}
	\end{columns}
	\vfill
}

\subsection{Tellur}
\frame{
	\frametitle{}
	\vfill
	\begin{itemize}
		\item Většina vyrobeného telluru se spotřebuje při výrobě ocelí a jiných slitin.
		\item Ocel s přídavkem telluru je snáze obrobitelná.
		\item Tellur v olovu zvyšuje jeho pevnost a zvyšuje odolnost vůči působení kyseliny sírové.\footnote[frame]{\href{https://doi.org/10.1016/j.jallcom.2008.08.011}{Study on the structure and property of lead tellurium alloy as the positive grid of lead-acid batteries}}
	\end{itemize}
	\begin{figure}
		\adjincludegraphics[height=0.4\textheight]{img/Car_battery_cross-section.jpg}
		\caption*{Řez olověným akumulátorem.\footnote[frame]{Zdroj: \href{https://commons.wikimedia.org/wiki/File:Car_battery_cross-section.jpeg}{Ben Cossalter/Commons}}}
	\end{figure}
	\vfill
}

\frame{
	\frametitle{}
	\vfill
	\begin{itemize}
		\item Významnou sloučeninou je tellurid kademnatý (\ce{CdTe}), má strukturu sfaleritu (ZnS).
		\item Využívá se při konstrukci solárních článků, výhodou jsou malé náklady. Tenkovrstvé CdTe články patří mezi levnější a jsou velmi rozšířené.
		\item Slitina s rtutí se využívá pro konstrukci MCT (Mercury Cadmium Telluride) detektorů pro infračervenou spektroskopii.\footnote[frame]{\href{http://irassociates.com/index.php?page=ln2-cooled}{LN2 Cooled HgCdTe Detectors}}
	\end{itemize}
	\begin{figure}
		\adjincludegraphics[height=0.35\textheight]{img/Photodetector.jpg}
		\caption*{Fotodetektor z CD-ROM mechaniky.\footnote[frame]{Zdroj: \href{https://commons.wikimedia.org/wiki/File:CD-ROM_Photodetector.jpg}{H0dges/Commons}}}
	\end{figure}
	\vfill
}

\frame{
	\frametitle{}
	\vfill
	\begin{columns}
		\begin{column}{.5\textwidth}
			\begin{figure}
				\adjincludegraphics[width=\textwidth]{img/TGIR2.jpg}
				\caption*{Otevřený TGIR modul s MCT detektorem.}
			\end{figure}
		\end{column}
		\begin{column}{.5\textwidth}
			\begin{figure}
				\adjincludegraphics[width=\textwidth]{img/TGIR3.jpg}
				\caption*{Detail MCT detektoru.}
			\end{figure}
		\end{column}
	\end{columns}
	\vfill
}

\frame{
	\frametitle{}
	\vfill
	\begin{figure}
		\adjincludegraphics[height=.68\textheight]{img/Sphalerite-unit-cell-depth-fade-3D-balls.png}
		\caption*{Krystalová struktura CdTe.\footnote[frame]{Zdroj: \href{https://commons.wikimedia.org/wiki/File:Sphalerite-unit-cell-depth-fade-3D-balls.png}{Benjah-bmm27/Commons}}}
	\end{figure}
	\vfill
}

\subsection{Polonium}
\frame{
	\frametitle{}
	\vfill
	\begin{itemize}
		\item Polonium slouží jako zdroj $\alpha$ částic, např. pro měření tloušťky pomocí absorpce částic.\footnote[frame]{HÁLA, Jiří. \textit{Radioaktivní izotopy}. Tišnov: Sursum, 2013. ISBN 978-80-7323-248-1.}
		\item \ce{^{210}_{\ 84}Po -> ^4_2He + ^{206}_{\ 82}Pb}
		\item T$_\frac{1}{2}$ = 138,376 dne
		\item Izotop $^{210}Po$ je vysoce toxický, právě díky vyzařování $\alpha$ částic. Rozpustná sůl $^{210}Po$ byla použita k otravě ruského agenta Alexandra Litviněnka.\footnote[frame]{\href{https://www.mirror.co.uk/news/uk-news/we-know-kgb-spy-poisoner-445700}{We know KGB spy poisoner}}
		\item Tento izotop bývá také označován za hlavní příčinu rakoviny plic u~kuřáků, protože se vyskytuje v tabákovém kouři.\footnote[frame]{\href{https://www.cdc.gov/nceh/radiation/polonium-210.htm}{Facts About Exposure to Polonium-210 from Naturally-Occurring Sources}}
		\item Dříve se využívala slitina s beryliem jako zdroj neutronů.
		\item V současnosti se využívá jako $\alpha$ zářič pro neutralizaci elektrostatického náboje.
	\end{itemize}
	\vfill
}

\section{Sloučeniny}
\subsection{Selenidy, telluridy a polonidy}
\frame{
	\frametitle{}
	\vfill
	\begin{itemize}
		\item Chakogenidy jsou sloučeniny chalkogenů s elektropozitivnějšími prvky.
		\item Jedná se o velice běžné minerály.
		\item Formálně se jsou to soli bezkyslíkatých kyselin: \ce{H2Se}, \ce{H2Te} a \ce{H2Po}.
		\item Mohou být jak stechiometrické, tak nestechiometrické.
		\item Selenidy a telluridy se připravují přímou reakcí prvků.
		\item Polonidy patří mezi nejstabilnější sloučeniny polonia.
		\item Polonidy lanthanoidů jsou stabilní až do teploty 1000~$^\circ$C.
		\item Reakcí roztoků alkalických kovů v kapalném amoniaku se selenem vznikají selenidy \ce{M2Se2}, \ce{M2Se3} a \ce{M2Se4}.
		\item Reakcí GaSe s kovovým cesiem vzniká lineární molekula \ce{Cs10Ga6Se14}.\footnote[frame]{\href{https://doi.org/10.1002/anie.198109621}{\ce{[Ga6Se14]$^{10-}$}: A 1900 pm Long, Hexameric Anion}}
	\end{itemize}
	\begin{figure}
		\adjincludegraphics[width=0.8\textwidth]{img/Ga6Se14.png}
	\end{figure}
	\vfill
}

\subsection{Hydridy}
\frame{
	\frametitle{}
	\vfill
	\begin{itemize}
		\item Selan, \ce{H2Se}, je bezbarvý, toxický, páchnoucí plyn.
		\item Připravuje se hydrolýzou selenidu hlinitého nebo rozkladem selenidu železnatého kyselinou chlorovodíkovou.
		\end{itemize}
		\begin{align*}
			\ce{Al2Se3 + 6 H2O &-> 3 H2Se + 2 Al(OH)3}\\
			\ce{FeSe + 2 HCl &-> H2Se + FeCl2}\\
			\ce{Se + H2 &<=>[520 $^\circ$C] H2Se}
		\end{align*}
		\begin{itemize}
		\item Při syntéze z prvků závisí výtěžek na teplotě, při vysoké teplotě dochází k rozkladu selanu.
		\item Optimální teplotou je 520~$^\circ$C.
		\item Na vzduchu hoří za vzniku \ce{SeO2}.
		\item \ce{2 H2Se + 3 O2 -> 2 SeO2 + 2 H2O}
	\end{itemize}
	\vfill
}

\frame{
	\frametitle{}
	\vfill
	\begin{itemize}
		\item Má podobné vlastnosti jako sulfan (p$K_a$ = 6,89), ale je kyselejší (p$K_a$ = 3,89).\footnote[frame]{\href{https://doi.org/10.1021/j100383a020}{Spectroscopic determination of the second dissociation constant of hydrogen selenide and the activity coefficients and spectral shifts of its ions}}
		\item Využívá se pro dopování polovodičů selenem.
		\item V organické syntéze se používá pro přípravu sloučenin obsahujících selen.\footnote[frame]{\href{https://www.thieme-connect.de/products/ejournals/pdf/10.1055/s-1980-28927.pdf}{A Convenient Synthesis of substituted Selenoureas from Methyl Carbamimidothioates (S-Methylpseudothioureas)}}
		\item \ce{R^{1}-N=C=N-R^2 + H2Se -> \chemfig{R^1-NH-C(=[6]Se)-NH-R^2}}
	\end{itemize}
	\vfill
}

\frame{
	\frametitle{}
	\vfill
	\begin{itemize}
		\item Tellan, \ce{H2Te}, je bezbarvý, toxický, páchnoucí plyn.
		\item Připravuje se elektrolýzou roztoku kyseliny sírové, katoda elektrolyzéru je z kovového telluru.
		\item Můžeme jej připravit také hydrolýzou \ce{Al2Te3}.
		\item \ce{Al2Te3 + 6 H2O -> 2 Al(OH)3 + 3 H2Te}
		\item Za laboratorní teploty se rozkládá, proto není možná příprava z prvků.
		\item \ce{H2Te ->[RT] Te + H2}
		\item Rozkládá se také působením vlhkého vzduchu a světla:
		\item \ce{2 H2Te + O2 ->[h$\nu$] 2 Te + 2 H2O}
		\item Na vzduchu hoří (podobně jako \ce{H2Se}) za vzniku \ce{TeO2}.
	\end{itemize}
	\vfill
}

\frame{
	\frametitle{}
	\vfill
	\begin{itemize}
		\item Polan, \ce{PoH2}, je těkavá, nestabilní kapalina.
		\item Na rozdíl od vody není kapalné skupenství způsobeno vodíkovými vazbami, ale van der Waalsovými interakcemi.
		\item Jeho příprava je komplikována možností radiolýzy.
		\item Příprava reakcí z prvků není proveditelná.
		\item Malá množství lze připravit rozpouštěním polonia na hořčíkové fólii ve zředěné HCl.
		\item Za vysokých tlaků vodíku (50--300 GPa) lze očekávat tvorbu dalších hydridových fází.\footnote[frame]{\href{https://doi.org/10.1039/C5RA19223D}{Prediction of stoichiometric \ce{PoH_n} compounds: crystal structures and properties)}}
	\end{itemize}
	\vfill
}

\subsection{Oxidy, hydroxidy a oxokyseliny}
\frame{
	\frametitle{}
	\vfill
	\begin{itemize}
		\item \textit{Oxid selenatý}, SeO, existuje pouze krátkodobě v plameni a není možné jej izolovat v pevném stavu.
		\item \textit{Oxid seleničitý}, \ce{SeO2}, je bílá pevná látka.
		\item Pevný oxid seleničitý je 1D polymer, ve kterém se střídají atomy Se a O.
	\end{itemize}
		\begin{figure}
		\adjincludegraphics[width=0.5\textwidth]{img/Selenium-dioxide-chain-3D-balls.png}
		\caption*{Struktura \ce{SeO2}.\footnote[frame]{Zdroj: \href{https://commons.wikimedia.org/wiki/File:Selenium-dioxide-chain-3D-balls.png}{Ben Mills/Commons}}}
	\end{figure}
	\begin{itemize}
		\item Lze jej připravit přímou reakcí z prvků nebo oxidací selenu pomocí peroxidu vodíku nebo kyseliny dusičné.
		\item Další možností je dehydratace kyseliny seleničité.
		\item Oxid seleničitý sublimuje, čehož lze využít k jeho čištění.
	\end{itemize}
	\vfill
}

\frame{
	\frametitle{}
	\vfill
	\begin{itemize}
		\item Ve vodě se rozpouští za vzniku kyseliny seleničité, s alkalickými hydroxidy poskytuje seleničitany.
		\item \ce{SeO2 + H2O -> H2SeO3}
		\item \ce{SeO2 + 2 NaOH -> Na2SeO3 + H2O}
		\item Oxid seleničitý se využívá jako oxidační činidlo v organické syntéze.\footnote[frame]{\href{http://www.adichemistry.com/organic/organicreagents/seo2/selenium-dioxide-seo2.html}{Selenium dioxide}}
		\item Dalším způsobem využití je barvení skla do červena, nebo k potlačování barvy skla způsobené železitými nečistotami.
	\end{itemize}
	\begin{figure}
		\adjincludegraphics[width=0.38\textwidth]{img/Seleniumdioxide_oxidation.png}
		\caption*{Oxidace pomocí \ce{SeO2}.\footnote[frame]{Zdroj: \href{https://commons.wikimedia.org/wiki/File:Seleniumdioxide_oxidation.svg}{Calvero/Commons}}}
	\end{figure}
	\vfill
}

\frame{
	\frametitle{}
	\vfill
	\begin{itemize}
		\item Kyselina seleničitá, \ce{H2SeO3}, je bílá pevná, hygroskopická látka.
		\item Lze ji připravit krystalizací vodného roztoku \ce{SeO2} nebo oxidací kovového selenu:
		\item \ce{3 Se + 4 HNO3 + H2O -> 3 H2SeO3 + 4 NO}
		\item Je to dvojsytná kyselina:
		\begin{itemize}
			\item p$K_a1$ = 2,62
			\item p$K_a2$ = 8,32
		\end{itemize}
	\item Využívá se k barvení (selenování) oceli na modrošedou až černou barvu.\footnote[frame]{\href{https://www.nrafamily.org/articles/2020/2/11/gun-manufacturing-browning-vs-bluing}{Gun Manufacturing: Browning vs. Bluing}}
	\end{itemize}
	\begin{figure}
		\adjincludegraphics[width=0.22\textwidth]{img/Selenous-acid-from-xtal-1971-3D-balls.png}
		\caption*{Kyselina seleničitá.\footnote[frame]{Zdroj: \href{https://commons.wikimedia.org/wiki/File:Selenous-acid-from-xtal-1971-3D-balls.png}{Ben Mills/Commons}}}
	\end{figure}
	\vfill
}

\frame{
	\frametitle{}
	\vfill
	\begin{itemize}
		\item Oxid selenový, \ce{SeO3}, je bílá pevná, hygroskopická látka.
		\item Není možné ho připravit přímou oxidací selenu. Nejčastěji se připravuje reakcí oxidu sírového se selenanem:
		\item \ce{SO3 + K2SeO4 -> K2SO4 + SeO3}
		\item V pevném stavu vytváří cyklické tetramerní molekuly (\ce{SO3} vytváří cyklické trimery).
		\item Stejně jako \ce{SO3} vytváří adukty s Lewisovými bazemi, např. pyridinem.
	\end{itemize}
	\begin{figure}
		\adjincludegraphics[width=0.7\textwidth]{img/SeO3-tetramer.png}
	\end{figure}
	\vfill
}

\frame{
	\frametitle{}
	\vfill
	\begin{columns}
		\begin{column}{.6\textwidth}
			\begin{itemize}
				\item Kyselina selenová, \ce{H2SeO4}, je bezbarvá, pevná, hygroskopická látka.
				\item Stejně jako kyselina sírová je velmi silnou kyselinou.
				\item Připravuje se oxidací oxidu seleničitého:
				\item \ce{SeO2 + H2O2 -> H2SeO4}
			\end{itemize}
		\end{column}
		\begin{column}{.4\textwidth}
			\begin{figure}
				\adjincludegraphics[width=\textwidth]{img/H2SeO4.png}
			\end{figure}
		\end{column}
	\end{columns}
	\begin{itemize}
	\setlength{\itemindent}{-1em}
		\item Lze ji připravit i oxidací selenu:
		\item \ce{Se + 3 Cl2 + 4 H2O -> H2SeO4 + 6 HCl}
		\item Dokáže rozpustit kovové zlato:\footnote[frame]{\href{https://doi.org/10.1021/ja02018a005}{Action of selenic acid on gold}}
		\item \ce{2 Au + 6 H2SeO4 ->[300 $^\circ$C] 3 SeO2 + Au2(SeO4)3 + 6 H2O}
		\item Při teplotách nad 200 $^\circ$C uvolňuje kyslík:
		\item \ce{2 H2SeO4 -> 2 H2SeO3 + O2}
	\end{itemize}
	\vfill
}

\frame{
	\frametitle{}
	\vfill
	\begin{itemize}
		\item \textit{Oxid tellurnatý}, \ce{TeO}, je nestabilní sloučenina, existující pouze krátký čas. Zatím se jej nepodařilo izolovat v čistém stavu.
		\item \textit{Oxid telluričitý}, \ce{TeO2}, existuje ve dvou polymorfních modifikacích.
		\item Syntetický $\alpha$-\ce{TeO2} tvoří tetragonální krystaly, složené z tetraedrů \ce{TeO4} propojených všemi vrcholy. Je bezbarvý.
		\item Připravuje se přímým slučování telluru s kyslíkem nebo dehydratací kyseliny kyseliny telluričité:
		\item \ce{H2TeO3 -> TeO2 + H2O}
		\item Přírodní $\beta$-\ce{TeO2} je rombický a má vrstevnatou strukturu.
		\item Oxid tellurový, \ce{TeO3}, je pevná látka.
		\item Vyskytuje se ve dvou formách:
		\begin{itemize}
			\item $\alpha$-\ce{TeO3} -- oranžový, struktura se skládá z oktaedrů \ce{TeO6}, které jsou propojeny vrcholy. Za vyšší teploty má silné oxidační účinky.
			\item $\beta$-\ce{TeO3} -- šedý, romboedrická struktura. Méně reaktivní než $\alpha$ modifikace.
		\end{itemize}
	\end{itemize}
	\vfill
}

\frame{
	\frametitle{}
	\vfill
	\begin{figure}
		\adjincludegraphics[height=0.65\textheight]{img/Cryst_struct_teo2.png}
		\caption*{Krystalová struktura \ce{TeO2}.\footnote[frame]{Zdroj: \href{https://commons.wikimedia.org/wiki/File:Cryst_struct_teo2.png}{Solid State/Commons}}}
	\end{figure}
	\vfill
}

\frame{
	\frametitle{}
	\vfill
	\begin{itemize}
		\item \textit{Kyselina tellurová}, \ce{H6TeO6}, je bílá, pevná látka.
		\item Má oktaedrickou geometrii, analog kyseliny sírové, \ce{H2TeO4}, nebyl dosud připraven.
		\item Připravuje se oxidací oxidu telluričitého nebo kovového telluru:
		\item \ce{TeO2 + H2O2 + 2 H2O -> H6TeO6}
		\item \ce{5 Te + 6 HClO3 + 12 H2O -> 5 H6TeO6 + 3 Cl2}
		\item Termickou dehydratací vzniká oxid tellurový.
		\item Má silné oxidační účinky:
		\item \ce{H6TeO6 + 3 SO2 -> Te + 3 H2SO4}
		\item \textit{Kyselina polymetatellurová}, \ce{(H2TeO4)_n}, je bílý, amorfní prášek.
		\item Vzniká částečnou dehydratací kyseliny tellurové a na vzduchu se zpět hydratuje.
		\item Od selenu i telluru známe i peroxokyseliny, např. kyselinu peroxoseleničitou (\ce{HOSeO(OOH)}).
	\end{itemize}
	\vfill
}

\frame{
	\frametitle{}
	\vfill
	\begin{itemize}
		\item \textit{Oxid polonatý}, \ce{PoO}, je černá pevná látka.
		\item Vzniká radiolýzou \ce{PoSO3}, při kontaktu s kyslíkem nebo vodou se snadno oxiduje na poloničité sloučeniny.
		\item \textit{Oxid poloničitý}, \ce{PoO2}, je žlutá krystalická látka. Krystaluje v kubické, plošně centrované soustavě (fluorit).
		\item Vzniká přímou oxidací polonia nebo termických rozkladem poloničitých solí.
		\item \ce{Po + O2 ->[250 $^\circ$C] PoO2}
		\item Reakcí s halogenovodíkem poskytuje odpovídající poloničité halogenidy.
		\item \ce{PoO2 + 4 HX -> PoX4 + 2 H2O}
		\item \textit{Oxid poloniový}, \ce{PoO3}, byl připraven pouze ve stopovém množství.
	\end{itemize}
	\vfill
}

\subsection{Sulfidy}
\frame{
	\frametitle{}
	\vfill
	\begin{itemize}
		\item Sulfidy selenu je možné připravit zahříváním selenu se sírou. Složení produktu pak závisí na poměru obou prvků.\footnote[frame]{\href{https://doi.org/10.1007/3-540-11345-2_11}{Cyclic selenium sulfides}}
		\item \ce{2 Se + 6 S -> Se2S6}
		\item Struktura je odvozena ze struktury \textit{cyklo}-oktasíry, kde je část atomů síry nahrazena atomy selenu.
		\item Jejich obecný vzorec je \ce{Se_nS_{8-n}}.
		\item Disulfid selenu se používá k léčbě kožních onemocnění a lupů.
	\end{itemize}

	\begin{columns}
		\begin{column}{.33\textwidth}
		\begin{figure}
			\adjincludegraphics[height=.3\textheight]{img/Se3S5.png}
			\caption*{\ce{1,3,5-Se3S5}}
		\end{figure}
		\end{column}

		\begin{column}{.33\textwidth}
			\begin{figure}
				\adjincludegraphics[height=.3\textheight]{img/Selenmonosulfide.png}
				\caption*{\ce{1,3,5,7-Se4S4}}
			\end{figure}
		\end{column}

		\begin{column}{.33\textwidth}
			\begin{figure}
				\adjincludegraphics[height=.3\textheight]{img/Selenhexasulfide.png}
				\caption*{\ce{1,2-Se2S6}}
			\end{figure}
		\end{column}
	\end{columns}
	\vfill
}

\frame{
	\frametitle{}
	\vfill
	\begin{itemize}
		\item Sulfid polonatý, PoS, je černá nerozpustná látka.
		\item Součin rozpustnosti je 28,3.
		\item Připravuje se reakcí chloridu se sulfanem nebo srážením hydroxidu sulfidem amonným:
		\item \ce{PoCl2 + H2S -> PoS + 2 HCl}
		\item \ce{(NH4)2S + Po(OH)2 -> PoS + 2 NH3 + 2 H2O}
		\item Vysoký součin rozpustnosti lze využít k odstraňování radioaktivního polonia z vody:\footnote[frame]{\href{https://doi.org/10.1134/S1066362215050148}{Use of Iron Sulfide for Removing Polonium from Liquid Radioactive Waste}}
		\item \ce{FeS + Po^{2+} -> PoS v + Fe^{2+}}
		\item Zahříváním dochází k rozkladu:
		\item \ce{PoS -> Po + S}
		\item Koncentrované kyseliny z něj uvolňují sulfan:
		\item \ce{PoS + 2 HCl -> PoCl2 + H2S ^}
	\end{itemize}
	\vfill
}

\subsection{Nitridy}
\frame{
	\frametitle{}
	\vfill
	\begin{itemize}
		\item Tetranitrid tetraselenu, \ce{Se4N4}, je oranžová látka, která velmi snadno exploduje (zahříváním nebo úderem).
		\item Lze jej připravit reakcí chloridu seleničitého s bis(trimethylsilyl)amidem lithným.\footnote[frame]{\href{https://doi.org/10.1021/ic00060a031}{A simple, efficient synthesis of tetraselenium tetranitride}}
		\item \small\ce{12 (Me3Si)2NLi + 2 Se2Cl2 + 8 SeCl4 -> 3 Se4N4 + 24 Me3SiCl + 12 LiCl}
		\item Je \textit{termochromní}:\footnote[frame]{Termochromní látky mění barvu s teplotou}
		\begin{itemize}
			\item Při teplotě $-$195 $^\circ$C je žlutooranžový.
			\item Při teplotě 100 $^\circ$C je červený.
		\end{itemize}
	\end{itemize}

	\begin{figure}
		\adjincludegraphics[width=.45\textwidth]{img/Se4N4.png}
	\end{figure}
	\vfill
}

\subsection{Halogenidy}
\frame{
	\frametitle{}
	\vfill
	\begin{itemize}
		\item V oxidačním čísle VI známe pouze fluoridy.
		\item V nižších oxidačních stavech je známo větší množství halogenidů.
		\item \ce{Se2F2} existuje ve dvou formách: \ce{FSe-SeF} a \ce{Se=SeF2}. Druhá forma vzniká při kondenzaci par za nízkých teplot.
		\item Jediným stabilním chloridem selenu je \ce{Se2Cl2}, ten obsahuje vazbu \ce{Se-Se}.
		\item Sloučeninu \ce{Te4I4} lze připravit reakcí telluru s nadbytkem jodu v křemenné trubici za teploty 950~$^\circ$C.\footnote[frame]{\href{https://doi.org/10.1107/S0108270191005656}{High‐temperature synthesis and structure redetermination of \ce{Te4I4}}} Reakční doba je sedm dnů, jodid tvoří černé jehlice.
		\item Monokrystalová XRD analýza prokázala, že v krystalu je každý atom telluru obklopen dalšími dvěma tellury a dvěma jodidy. Tellury mají čtvercově planární geometrii, nevazebné elektronové páry jsou v kolmé rovině.
		\item Fluorid tellurnatý není dosud znám.
	\end{itemize}
	\vfill
}

\frame{
	\frametitle{}
	\vfill
	\begin{figure}
		\adjincludegraphics[width=.8\textwidth]{img/Te4I4.png}
		\caption*{Krystalová struktura \ce{Te4I4}}
	\end{figure}
	\vfill
}

\frame{
	\frametitle{}
	\vfill
	\begin{itemize}
		\item \textbf{Fluorid seleničitý}, \ce{SeF4}, je bezbarvá, reaktivní kapalina.
		\item Krystaluje za vzniku bílé, hygroskopické pevné látky.
		\item Připravuje se fluorací selenu:
		\item \ce{Se + 2 F2 -> SeF4}
		\item \ce{3 Se + 4 ClF3 -> 3 SeF4 + 2 Cl2}
		\item Další možností je fluorace oxidu seleničitého pomocí \ce{SF4}:\footnote[frame]{\href{https://doi.org/10.1002/9780470132555.ch9}{Selenium Tetrafluoride, Selenium Difluoride Oxide (Seleninyl Fluoride), and Xenon Bis[Pentafluorooxoselenate(VI)]}}
		\item \ce{SeO2 + SF4 -> SeF4 + SO2}
		\item Využívá se jako fluorační činidlo.
		\item V souladu s teorií VSEPR má molekula tvar houpačky.
		\item Reakcí s CsF poskytuje pentafluoroseleničitany:\footnote[frame]{\href{https://doi.org/10.1021/ic50113a046}{Vibrational spectra nad force constants of the square-pyramidal anions SF$_5^-$, SeF$_5^-$, and TeF$_5^-$}}
		\item \ce{SeF4 + CsF -> Cs[SeF5]}
	\end{itemize}
	\vfill
}

\frame{
	\frametitle{}
	\vfill
	\begin{itemize}
		\item \textbf{Chlorid seleničitý}, \ce{SeCl4}, je žlutá pevná látka. Sublimuje při teplotě 191~$^\circ$C.
		\item Připravuje se přímou chlorací selenu, produkt lze izolovat sublimací. Těkavosti této sloučeniny lze využít k čištění selenu.
		\item Na rozdíl od fluoridu neodpovídá tvar molekuly teorii VSEPR.
		\item Chlorid seleničitý vytváří tetramerní, kubické molekuly. Ty jsou tvořeny oktaedry \ce{SeCl6}.\footnote[frame]{\href{https://dx.doi.org/10.1515/znb-1981-1231}{Crystal Structure of the Stable Modification of \ce{SeCl4}}}
		\item Hydrolýzou vzniká kyselina seleničitá:\footnote[frame]{\href{https://www.chemicalpapers.com/?id=7&paper=7448}{Synthesis of pure selenium tetrachloride and its hydrolysis to selenium oxychloride}}
		\item \ce{SeCl4 + 3 H2O -> H2SeO3 + 4 HCl}
		\item V prostředí koncentrované HCl poskytuje s chloridy alkalických kovů komplexní ionty:
		\item \ce{SeCl4 + 2 KCl -> K2[SeCl6]}
	\end{itemize}
	\vfill
}

\frame{
	\frametitle{}
	\vfill
	\begin{figure}
		\adjincludegraphics[height=.7\textheight]{img/SeCl4-xtal.png}
		\caption*{Kubická jednotka chloridu seleničitého.\footnote[frame]{\href{https://commons.wikimedia.org/wiki/File:SeCl4-from-alpha-xtal-1981-CM-3D-ellipsoids.png}{Zdroj: Ben Mills/Commons}}}
	\end{figure}
	\vfill
}

\frame{
	\frametitle{}
	\vfill
	\begin{itemize}
		\item \textbf{Fluorid telluričitý}, \ce{TeF4}, je bílá, hygroskopická látka (T$_t$ = 129~$^\circ$C).
		\item Připravuje se fluorací \ce{TeO2}, příp. opatrnou fluorací telluru nebo tellurnatých sloučenin směsí fluoru a dusíku.
		\item \ce{TeO2 + 2 SF4 -> TeF4 + 2 SOF2}
		\item \ce{Te + 2 F2 ->[N2] TeF4}
		\item Připravený fluorid lze vyčistit vakuovou sublimací při 100~$^\circ$C.
		\item V plynné fázi je monomerní.
		\item V krystalickém stavu vytváří řetězce tvořené tetragonálními pyramidami \ce{TeF5}, ty jsou propojeny můstkovými fluoridy v poloze \textit{cis}.\footnote[frame]{\href{https://doi.org/10.1039/J19680002977}{Fluoride crystal structures. Part IV. Tellurium tetrafluoride}}
		\item Úhel \ce{Te-F-Te} je 159$^\circ$.
		\item Volná pozice oktaedru je obsazena nevazebným elektronovým párem.
	\end{itemize}
	\vfill
}

\frame{
	\frametitle{}
	\vfill
	\begin{figure}
		\adjincludegraphics[height=.7\textheight]{img/TeF4.png}
		\caption*{Krystalová struktura \ce{TeF4}.\footnote[frame]{\href{https://en.wikipedia.org/wiki/File:Tellurium-tetrafluoride-xtal-1984-3D-balls.png}{Zdroj: Ben Mills/Commons}}}
	\end{figure}
	\vfill
}

\subsection{Organokovové sloučeniny selenu}
\frame{
	\frametitle{}
	\vfill
	\begin{itemize}
		\item Oxidační číslo selenu je převážně II a selen nese dva nevazebné elektronové páry.
		\item Sloučeniny jsou nukleofilnější a kyselejší než odpovídající sloučeniny síry.
		\item První připravenou organokovovou sloučeninou selenu byl diethylselan (\ce{Et2Se}).\footnote[frame]{\href{https://doi.org/10.1021/cr900352j}{Organoselenium Chemistry: Role of Intramolecular Interactions}}
	\end{itemize}
	\begin{columns}
	\begin{column}{.4\textwidth}
		\begin{tabular}{|l|l|l|l|}
		\hline
		XH & \ce{H2O} & \ce{H2S} & \ce{H2Se} \\\hline
		p$K_a$ & 14 & 7 & 3,8 \\\hline
	\end{tabular}
	\end{column}
	\begin{column}{.6\textwidth}
	\begin{figure}
	\adjincludegraphics[width=\textwidth]{img/organoselenium.png}
	\end{figure}
	\end{column}
	\end{columns}
	\vfill
}

\subsection{Organokovové sloučeniny telluru}
\frame{
	\frametitle{}
	\vfill
	\begin{itemize}
		\item Struktury organokovových sloučenin telluru jsou podobné jako v~případě selenu.
		\item Běžnými reagenty jsou tellan, hydrogentellurid sodný a fenyltellurid lithný (\ce{PhTeLi}).
		\item Kovový tellur je nerozpustný, proto není ideální výchozí látkou. Reaguje ale s komplexními hydridy:
		\item \ce{Te + 2 LiBHEt3 -> Li2Te + H2 + 2 Et3B}
		\item nebo s organolitnými sloučeninami:
		\item \ce{Te + PhLi -> PhTeLi}
		\item Dimethyltellurid se používá jako těkavý prekurzor Te.
		\item \ce{Me2Te}, stejně jako \ce{Me2Se} tvoří ochotně adukty:\footnote[frame]{\href{https://doi.org/10.1016/j.jorganchem.2017.08.004}{Complexes of \ce{BX3} with \ce{EMe2}}}
		\item \ce{Me2Te + BCl3 ->[hexan] Me2Te.BCl3}
		\item \ce{Me2Te + BF3 ->[CH2Cl2] Me2Te.BF3}
	\end{itemize}
	\vfill
}

\subsection{Organokovové sloučeniny polonia}
\frame{
	\frametitle{}
	\vfill
	\begin{itemize}
		\item U polonia jsou znalosti o organokovových sloučeninách menší.\footnote[frame]{\href{https://doi.org/10.1002/9781119951438.eibc0182}{Polonium: Organometallic Chemistry}}
		\item Problémem je nejen radiolýza organokovových sloučenin, ale i problémy se získáním dostatečného množství kovového polonia.
		\item Známe následující typy sloučenin:
		\begin{itemize}
			\item Dialkyl a diarylpolonidy
			\item Halogenidy triarylpolonia (\ce{Ar3PoCl})
			\item Dihalogenidy diarylpolonia (\ce{Ar2PoCl2})
			\item Diarylpoloniumoxidy (\ce{(C9H11)2Po=O})
		\end{itemize}
	\end{itemize}
	\begin{figure}
		\adjincludegraphics[height=.35\textheight]{img/Ar2PoO.png}
	\end{figure}
	\vfill
}

\section{Biologie}
\frame{
	\frametitle{}
	\vfill
	\begin{columns}
		\begin{column}{.7\textwidth}
			\begin{itemize}
				\item Selen je větším množství toxický, ale ve stopovém množství je pro živočichy nezbytný.\footnote[frame]{\href{https://dx.doi.org/10.1001/archinternmed.2009.495}{Acute Selenium Toxicity Associated With a Dietary Supplement}}
				\item Je součástí aminokyselin selenocysteinu a selenomethioninu.
				\item Komerčně jsou dostupné doplňky stravy obsahující selen.\footnote[frame]{\href{https://www.bezpecnostpotravin.cz/selen-zdroje-ucinky-a-zasobovani.aspx}{Selen – zdroje, účinky a zásobování}}
				\item Doporučená denní dávka selenu pro člověka je 1~mg.kg$^{-1}$.\footnote[frame]{\href{https://www.wikiskripta.eu/w/Selen}{Selen}}
				\item Přirozeným zdrojem selenu jsou cereálie a~mořské produkty.
				\item Otravy selenem jsou vzácné, akutní otrava se projevuje česnekovým zápachem potu a z úst. Chronická vypadáváním vlasů a nehtů.
			\end{itemize}
		\end{column}
		\begin{column}{.35\textwidth}
			\begin{figure}
				\adjincludegraphics[height=.75\textheight]{img/Se-AminoAcids.png}
			\end{figure}
		\end{column}
	\end{columns}
	\vfill
}

\frame{
	\frametitle{}
	\vfill
	\begin{figure}
		\adjincludegraphics[height=0.2\textheight]{img/ebselen.png}
	\end{figure}
	\begin{itemize}
		\item Syntetické léčivo \textit{ebselen} má anti-oxidační účinky a zdá se být slibným léčivem proti COVID-19.\footnote[frame]{\href{https://www.nature.com/articles/s41586-020-2223-y}{Structure of Mpro from SARS-CoV-2 and discovery of its inhibitors}}
		\item Syntéza ebselenu a jeho derivátů probíhá podle schématu:\footnote[frame]{\href{https://doi.org/10.1002/hc.21164}{Synthesis and Antioxidant Activities of Novel Chiral Ebselen Analogues}}
	\end{itemize}
	\begin{figure}
		\adjincludegraphics[width=\textwidth]{img/Ebselen_Routes.png}
	\end{figure}
	\vfill
}

\frame{
	\frametitle{}
	\vfill
	\begin{columns}
	\begin{column}{.55\textwidth}
		\begin{itemize}
			\item Tellur není příliš rozšířený v biologických systémech a jeho toxikologie není dosud příliš prozkoumaná.\footnote[frame]{\href{https://doi.org/10.1007/978-1-4614-1533-6_504}{Tellurium in Nature}}
			\item Některé houby (např. \textit{Aspergillus fumigatus} a \textit{Aspergillus terreus}) dokáží místo síry využívat tellur.\footnote[frame]{\href{https://doi.org/10.1007/BF02917437}{Incorporation of tellurium into amino acids and proteins in a tellurium-tolerant fungi}}
		\end{itemize}
	\end{column}
	\begin{column}{.5\textwidth}
		\begin{figure}
		\adjincludegraphics[width=\textwidth]{img/Aspergillus_on_tomato.jpg}
		\caption{Plíseň Aspergillus na rajčeti.\footnote[frame]{\href{https://commons.wikimedia.org/wiki/File:Aspergillus_on_tomato.jpg}{Zdroj: Multimotyl/Commons}}}
		\end{figure}
	\end{column}
	\end{columns}
	\vfill
}

\input{../Last}

\end{document}